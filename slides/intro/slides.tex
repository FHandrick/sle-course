\documentclass{beamer}

\title{Rascal-MPL \\ A Short Introduction}
\subtitle{Slides Based on the P. Klint's Introductory Course on Rascal}

\author{Rodrigo Bonif\'{a}cio}

\begin{document}

\begin{frame}
\titlepage
\end{frame}

\begin{frame}
  \begin{quote}
Rascal is an experimental domain specific language for metaprogramming, such as static code analysis, program transformation and implementation of domain specific languages. It is a general meta language in the sense that it does not have a bias for any particular software language.
  \end{quote}
  \flushright{Wikipedia}
\end{frame}

\begin{frame}
  \begin{block}{Domain}
    Rascal-MPL allows a developer
    to create full-fledged program
    analysis and manipulation tools,
    including support for syntax
    definition and strategic programming. 
  \end{block}
\end{frame}

\begin{frame}
  \frametitle{Features}

  \begin{itemize}
    \item hybrid programming model (imperative and functional styles)
    \item immutable data structures and support for generic types
    \item pattern matching and visitors using different strategies
    \item syntax definitions and parsing + REPL support  
    \item Java and Eclipse integration  
  \end{itemize}  
\end{frame}

\begin{frame}[fragile]
  \frametitle{ColoredTrees}

  \begin{small}
  \begin{verbatim}
module ColoredTrees

data CTree = leaf(int N)
           | red(CTree left, CTree right)
           | black(CTree left, CTree right)
           ;


CTree sample() = red(black(leaf(1), red(leaf(2), leaf(3))),
                     black(leaf(4), leaf(5))); 
\end{verbatim}
  \end{small}
\end{frame}

\begin{frame}[fragile]
  \frametitle{Pattern Matching (switch-case)}

  \begin{block}{Number of elements of a tree}
    \begin{small}
    \begin{verbatim}
    int elements(CTree tree) {
      int res = 0;
      switch(tree) {
        case leaf(n): res = 1;
        case red(l, r): res = 1 + elements(l) + elements(r);
        case black(l, r):  res = 1 + elements(l) + elements(r);
      }
      return res;
    }
\end{verbatim}
    \end{small}xs
  \end{block}  
  
\end{frame}

\begin{frame}[fragile]
  \frametitle{Pattern Matching (visitor to collect information)}

  \begin{block}{Total of values in a tree}
    \begin{small}
\begin{verbatim}
    int sumTree(CTree tree) {
      int res = 0;
      top-down visit (tree) {
        case leaf(n): res = res + n;
      }
      return res;
    }
\end{verbatim}
\end{small}
  \end{block}\pause

  \begin{itemize}
    \item Traversal strategies
      \begin{itemize}
        \item top-down visit (top-down-break)
        \item bottom-up visit (bottom-up-break)
        \item innermost
        \item outermost  
      \end{itemize}  
  \end{itemize}  
\end{frame}

\begin{frame}[fragile]
  \frametitle{Pattern Matching (visitor to transform a tree)}

  \begin{block}{Increment all leaves}
    \begin{verbatim}
    CTree increment(CTree tree) =
      top-down visit (tree) {
        case leaf(n) => leaf(n + 1)
      };
    \end{verbatim}  
  \end{block}\pause

  \begin{block}{Change first black to red}
    \begin{verbatim}
    CTree changeFirstBlack(CTree tree) =
      top-down-break visit (tree) {
        case black(l, r) => red(l, r)
      };
    \end{verbatim}  
  \end{block}\pause

\end{frame}


\begin{frame}[fragile]
  \frametitle{Comprehension}

  \begin{verbatim}
    GHCi, version 7.10.3: http://www.haskell.org/ghc/  :? for help
    Prelude> let list = [1..10]
    Prelude> [x * x | x <- list, even x]
    [4,16,36,64,100]    
  \end{verbatim}
  
  Comprehension is available in many programming languages\pause,
  including Haskell, Python, Rascal, and so on.

  \begin{block}{Rascal combines comprehension and generic traversals}
   \begin{verbatim}
     rascal> [n | /leaf(int n) <- sample()];
     list[int]: [1,2,3,4,5]
   \end{verbatim}
  \end{block}
  
\end{frame}

\begin{frame}[fragile]
  \frametitle{Export to DOT: String Interpolation (1/3)}

    \begin{verbatim}
    str export(CTree t) {
      str g = ``digraph g { \n'';
        int id = 0;
        map[CTree, str] decls = ();
\end{verbatim}

\end{frame}

\begin{frame}[fragile]
  \frametitle{Export to DOT: String Interpolation (2/3)}

  \begin{small}
    \begin{verbatim}
        // print the nodes
        top-down visit(t) {
          case red(l, r) :
          {
            id = id + 1;
            g += ``  n<id> [label = \"\" shape = circle fillcolor = red style = filled]\n'';
            decls += (red(l,r) : ``n<id>'');
          }
          case black(l, r) :
          {
            id = id + 1;
            g += ``  n<id> [label = \"\" shape = circle fillcolor = black style = filled]\n'';
            decls += (black(l,r) : ``n<id>'');
          }
          case leaf(n) :
          {
            id = id + 1;
            g = g + ``  n<id> [label = <n>] \n'';
            decls += (leaf(n) : ``n<id>'');
          }
        };
\end{verbatim}
  \end{small}
\end{frame}

\begin{frame}[fragile]
  \frametitle{Export to DOT: String Interpolation (3/3)}
  \begin{small}
    \begin{verbatim}
        // print the edges
        top-down visit(t) {
          case red(l, r) : {
            g += ``  <decls[red(l,r)]> -\> <decls[l]> \n'';
            g += ``  <decls[red(l,r)]> -\> <decls[r]> \n'';
          }
          case black(l, r) : {
            g += ``  <decls[black(l,r)]> -\> <decls[l]> \n'';
            g += ``  <decls[black(l,r)]> -\> <decls[r]> \n'';
          }
        }

        g += ``}'';

      return g;
    }
  \end{verbatim}  
  \end{small}  
\end{frame}

\end{document}
