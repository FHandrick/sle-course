\documentclass{beamer}

\usepackage{lstcustom} 

\lstdefinelanguage{Rascal}[]{Java}{
  morekeywords={module, syntax, start, data, visitor},
  moredelim=[is][\textcolor{darkgray}]{\%\%}{\%\%},
  moredelim=[il][\textcolor{darkgray}]{§§}
}

\lstdefinelanguage{FSM}[]{Java}{
  morekeywords={initial, state},
  moredelim=[is][\textcolor{darkgray}]{\%\%}{\%\%},
  moredelim=[il][\textcolor{darkgray}]{§§}
}

\title{Rascal-MPL \\ A Short Introduction}
\subtitle{Slides Based on the P. Klint's Introductory Course on Rascal}

\author{Rodrigo Bonif\'{a}cio}

\begin{document}

\begin{frame}
\titlepage
\end{frame}

\begin{frame}
  \begin{quote}
Rascal is an experimental domain specific language for metaprogramming, such as static code analysis, program transformation and implementation of domain specific languages. It is a general meta language in the sense that it does not have a bias for any particular software language.
  \end{quote}
  \flushright{Wikipedia}
\end{frame}

\begin{frame}
  \begin{block}{Domain}
    Rascal-MPL allows a developer
    to create full-fledged program
    analysis and manipulation tools,
    including support for syntax
    definition and strategic programming. 
  \end{block}
\end{frame}

\begin{frame}
  \frametitle{Features}

  \begin{itemize}
    \item hybrid programming model (imperative and functional styles)
    \item immutable data structures and support for generic types
    \item pattern matching and visitors using different strategies
    \item syntax definitions and parsing + REPL support  
    \item Java and Eclipse integration  
  \end{itemize}  
\end{frame}

\begin{frame}[fragile]
  \frametitle{ColoredTrees}

  \begin{small}
  \begin{lstlisting}[language=Rascal]
module ColoredTrees

data CTree = leaf(int N)
           | red(CTree left, CTree right)
           | black(CTree left, CTree right)
           ;


CTree sample() = red(black(leaf(1), red(leaf(2), leaf(3))),
                     black(leaf(4), leaf(5))); 
\end{lstlisting}
  \end{small}
\end{frame}

\begin{frame}[fragile]
  \frametitle{Pattern Matching (switch-case)}

  \begin{block}{Number of elements of a tree}
    \begin{small}
    \begin{lstlisting}[language=Rascal]
    int elements(CTree tree) {
      int res = 0;
      switch(tree) {
        case leaf(n): res = 1;
        case red(l, r): res = 1 + elements(l) + elements(r);
        case black(l, r):  res = 1 + elements(l) + elements(r);
      }
      return res;
    }
\end{lstlisting}
    \end{small}xs
  \end{block}  
  
\end{frame}

\begin{frame}[fragile]
  \frametitle{Pattern Matching (visitor to collect information)}

  \begin{block}{Total of values in a tree}
    \begin{small}
\begin{lstlisting}[language=Rascal]
    int sumTree(CTree tree) {
      int res = 0;
      top-down visit (tree) {
        case leaf(n): res = res + n;
      }
      return res;
    }
\end{lstlisting}
\end{small}
  \end{block}\pause

  \begin{itemize}
    \item Traversal strategies
      \begin{itemize}
        \item top-down visit (top-down-break)
        \item bottom-up visit (bottom-up-break)
        \item innermost
        \item outermost  
      \end{itemize}  
  \end{itemize}  
\end{frame}

\begin{frame}[fragile]
  \frametitle{Pattern Matching (visitor to transform a tree)}

  \begin{block}{Increment all leaves}
    \begin{lstlisting}[language=Rascal]
    CTree increment(CTree tree) =
      top-down visit (tree) {
        case leaf(n) => leaf(n + 1)
      };
    \end{lstlisting}  
  \end{block}\pause

  \begin{block}{Change first black to red}
    \begin{lstlisting}[language=Rascal]
    CTree changeFirstBlack(CTree tree) =
      top-down-break visit (tree) {
        case black(l, r) => red(l, r)
      };
    \end{lstlisting}  
  \end{block}\pause

\end{frame}


\begin{frame}[fragile]
  \frametitle{Comprehension}

  \begin{lstlisting}[language=Rascal]
    GHCi, version 7.10.3: http://www.haskell.org/ghc/  :? for help
    Prelude> let list = [1..10]
    Prelude> [x * x | x <- list, even x]
    [4,16,36,64,100]    
  \end{lstlisting}
  
  Comprehension is available in many programming languages\pause,
  including Haskell, Python, Rascal, and so on.

  \begin{block}{Rascal combines comprehension and generic traversals}
   \begin{lstlisting}[language=Rascal]
     rascal> [n | /leaf(int n) <- sample()];
     list[int]: [1,2,3,4,5]
   \end{lstlisting}
  \end{block}
  
\end{frame}

\begin{frame}[fragile]
  \frametitle{Export to DOT: String Interpolation (1/3)}

    \begin{lstlisting}[language=Rascal]
    str export(CTree t) {
      str g = "digraph g { \n";
        int id = 0;
        map[CTree, str] decls = ();
\end{lstlisting}

\end{frame}

\begin{frame}[fragile]
  \frametitle{Export to DOT: String Interpolation (2/3)}

  \begin{small}
    \begin{lstlisting}[language=Rascal]
        // print the nodes
        top-down visit(t) {
          case red(l, r) :
          {
            id = id + 1;
            g += "  n<id> [label = \"\" shape = circle fillcolor = red style = filled]\n";
            decls += (red(l,r) : "n<id>");
          }
          case black(l, r) :
          {
            id = id + 1;
            g += "  n<id> [label = \"\" shape = circle fillcolor = black style = filled]\n";
            decls += (black(l,r) : "n<id>");
          }
          case leaf(n) :
          {
            id = id + 1;
            g = g + "  n<id> [label = <n>] \n";
            decls += (leaf(n) : "n<id>");
          }
        };
\end{lstlisting}
  \end{small}
\end{frame}

\begin{frame}[fragile]
  \frametitle{Export to DOT: String Interpolation (3/3)}
  \begin{small}
    \begin{lstlisting}[language=Rascal]
        // print the edges
        top-down visit(t) {
          case red(l, r) : {
            g += "  <decls[red(l,r)]> -\> <decls[l]> \n";
            g += "  <decls[red(l,r)]> -\> <decls[r]> \n";
          }
          case black(l, r) : {
            g += "  <decls[black(l,r)]> -\> <decls[l]> \n";
            g += "  <decls[black(l,r)]> -\> <decls[r]> \n";
          }
        }

        g += "}";

      return g;
    }
  \end{lstlisting}  
  \end{small}  
\end{frame}


\begin{frame}
\begin{huge}
Syntax Definition
\end{huge}
\pause 
\vskip+1.5em
\begin{itemize}
\item converts a grammar into a SGLR parser
\item main components (context free grammar) 
\begin{itemize}
  \item syntax expressed using production rules
  \item lexical, layout, and keyword definitions
\end{itemize} \pause
\item challenge: disambiguation
\end{itemize}
\end{frame}

\begin{frame}[fragile]
\frametitle{Let's consider a language for FSMs} 

\begin{lstlisting}[language=Rascal]
module lang::fsm::AbstractSyntax

data StateMachine 
  = fsm(list[State] states, list[Transition] transitions);

data State 
  = state(str name)
  | startState(str name)
  ;
           
data Transition 
  = transition(str source, Event event, str target);
  
data Event 
  = event(str evt)
  | eventWithAction(str evt, str action)
  ;
\end{lstlisting}
\end{frame}

\begin{frame}[fragile]
 \frametitle{Example of a FSM specification}

 \begin{lstlisting}[language=FSM]
initial state locked {
  ticket/collect -> unlocked;
  pass/alarm -> exception;
}

state unlocked {
  ticket/eject;
  pass -> locked;
}

state exception {
  ticket/eject;
  pass;
  mute;
  release -> locked;
}
 \end{lstlisting}
\end{frame}

\begin{frame}[fragile]
\frametitle{A grammar definition for FSMs using Rascal}

\begin{lstlisting}[language=Rascal]
module lang::fsm::ConcreteSyntax

start syntax FSM = CState* states;

syntax CState 
  = initialState: "initial" "state" Id "{" CEvent* "}"
  | basicState: "state" Id "{" CEvent* "}"
  ;

syntax CEvent 
  = basicEvent: Id ("-\>" Id)? ";"
  | actionEvent: Id "/" Id ("-\>" Id)? ";"
  ; 

lexical Id = [a-zA-Z][_a-zA-Z0-9]*; 

lexical Comment = "//" ![\n]* [\n];

lexical Spaces = [\n\r\f\t\ ]*;

layout Layout = Spaces 
              | Comment 
              ; 

keyword Keywords = "state" | "initial" ;
\end{lstlisting} 

\end{frame}

\begin{frame}[fragile]
\frametitle{A parser from Concrete Syntax to Abstract Syntax (1/2)}

\begin{lstlisting}[language=Rascal]
module lang::fsm::Parser

import ParseTree;
import IO;

import lang::fsm::ConcreteSyntax;
import lang::fsm::AbstractSyntax; 

public StateMachine parseFSM(str f) {
 loc file = |file:///| + f;
 list[State] states = [];
 list[Transition] transitions = [];
 
 start[FSM] parseResult = parse(#start[FSM], file);
 
 top-down visit (parseResult) {
    case (CState)`initial state <Id id> { <CEvent* evts>}`: {
      states += startState(unparse(id));
      transitions += parseEvents(id, evts);
    }
    case (CState)`state <Id id> { <CEvent* evts>}`: {
      states += state(unparse(id));
      transitions += parseEvents(id, evts);
    }
 };
 
 return fsm(states, transitions);
}
\end{lstlisting}

\end{frame}

\begin{frame}[fragile]
\frametitle{A parser from Concrete Syntax to Abstract Syntax (2/2)}

\begin{lstlisting}[language=Rascal]
list[Transition] parseEvents(Id source, CEvent* evts) {
  list[Transition] ts = [];
  top-down visit(evts) {
    case (CEvent)`<Id e> -\> <Id target>;` : 
      ts += transition(unparse(source), event(unparse(e)), unparse(target));
    case (CEvent)`<Id e>;` : 
      ts += transition(unparse(source), event(unparse(e)), unparse(source));
    case (CEvent)`<Id e> / <Id a> -\> <Id target>;` : 
      ts += transition(unparse(source), eventWithAction(unparse(e), unparse(a)), unparse(target));
    case (CEvent)`<Id e> / <Id a>;` : 
      ts += transition(unparse(source), eventWithAction(unparse(e), unparse(a)), unparse(source));
  }
  return ts;
}
\end{lstlisting}

\end{frame}

\end{document}
